\section{Introduction}
\label{sec-introduction}

This abstract examines the behavior of the participants in \PhoneLab{}, a public
smartphone testbed being developed at SUNY Buffalo. Currently consisting of 191
participants using Nexus S 4G smartphones, \PhoneLab{} aims to provide a
combination of unique features desirable for smartphone experimentation. This
abstract briefly introduces \PhoneLab{} and presents some of the early results
of a usage measurement study conducted with 115 participants. 

\subsection{\PhoneLab{} Overview}

\PhoneLab{} is designed to provide the following features necessary for
smartphone research---open access, scale, power, realism, locality, and
relevance:

\begin{itemize}
\itemsep -2pt
\item {\bf Open Access:} After the initial approval process, \PhoneLab{} allows
any researcher to deploy their research prototype on the participants'
smartphones.
\item {\bf Scale:} By 2014, \PhoneLab{} will grow to 700 participants already
incentivized and recruited to participate in experiments; participants of
\PhoneLab{} receive discounted voice, data, and messaging.
\item {\bf Power:} By utilizing the Android open-source smartphone platform,
\PhoneLab{} allows application-level experiments as well as platform-level,
i.e., the OS kernel, middleware, and libraries.
\item {\bf Realism:} Participants use the phones as their primary device.
\item {\bf Locality:} Most participants live in Buffalo near SUNY campuses,
enabling research requiring device-to-device interaction.
\item {\bf Relevance:} \PhoneLab{} allows researchers to stop relying on
out-of-date datasets. Instead, new data can be collected in the most
appropriate way for the experiment.
\end{itemize}

\PhoneLab{} application-level experiments are distributed through the Play
Store; participants are notified of new experiments and install the experimental
applications directly from the Play Store. On the other hand, \PhoneLab{}
platform-level experiments are distributed through the \PhoneLab{} control
software that runs on each participant's phone; this control software is capable
of updating platform components, e.g., libraries and kernel modules. To the best
of our knowledge, \PhoneLab{} is the only testbed that provides all the above
features together.

\subsection{\PhoneLab{} Demographics}

Currently, \PhoneLab{} consists of 191 participants. Roughly half of our
participants are first- and second-year undergraduates, a quarter PhD students,
and a fifth faculty, staff and other professionals. However, males greatly
outnumber females, and the young outnumber the middle-aged and older, both
unrepresentative features we will try and rectify in the future years. For
management reasons we limited participation to people with a SUNY Buffalo
affiliation except for several exceptions: a local reporter, a technology
writer, and an international rock star. Table~\ref{tab:demographics} summarizes
our demographics.

\begin{table}[t]
\begin{threeparttable}
\begin{tabularx}{\columnwidth}{Xr@{\hspace{0.5in}}Xr}
\multicolumn{4}{c}{\textbf{Affiliation}} \\
\midrule
Freshman & 64 & Masters & 5 \\
Sophomore & 33 & PhD & 53 \\
Junior & 1 & Faculty/Staff & 29 \\
Senior & 1 & None & 5 \\[0.1in]
\multicolumn{4}{c}{\textbf{Gender}} \\
\midrule
Female & 51 & Male & 140 \\[0.1in]
\multicolumn{4}{c}{\textbf{Age}} \\
\midrule
Under 18 & 12 & 30--34 & 15 \\
18--19 & 74 & 35--39 & 6 \\
20--21 & 12 & 40--49 & 13 \\
22--24 & 22 & 50--59 & 7 \\
25--29 & 29 & 60+ & 1\\
\end{tabularx}
\end{threeparttable}
\caption{Demographic breakdown of 191 \PhoneLab{} participants. Date
ranges are inclusive.}
\vspace{-.2in}
\label{tab:demographics}
\end{table}
