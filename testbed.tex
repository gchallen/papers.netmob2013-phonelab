\section{The PhoneLab Testbed}
\label{sec-testbed}

\PhoneLab{} was designed to fill a gap in existing smartphone experimentation
capabilities.\PhoneLab{} achieves power by utilizing Android open-source devices and a self-signed build which allows us to update any software components; scale by amortizing recruitment overhead, management burden and incentive costs across multiple experiments; and realism by recruiting a diverse set of participants and limiting experimental intrusiveness. We describe the architecture \PhoneLab{} in more detail below.

\subsection{Overview}

\PhoneLab{} currently consists of 191 participants\footnote{We refer to
people carrying \PhoneLab{} phones and participating in experiments as
\PhoneLab{} \textit{participants}, to differentiate them from researchers
running \PhoneLab{} experiments which we call \textit{users}.} using Sprint
Nexus~S~4G smartphones~\cite{nexuss4g} running Android 4.1.1, Jelly
Bean~\cite{jellybean}. Participants receive discounted voice, data, and
messaging, and are instructed to use their \PhoneLab{} phone as their primary
device.

\PhoneLab{} experiments are either distributed through the Play Store or as
platform over-the-air (OTA) updates. Participants are notified of new
experiments and choose whether to participate after reviewing what
information will be collected about them. \PhoneLab{} participants are
\textit{required} to participate in experimentation but \textit{not required}
to participate in any particular experiment. Experiments generate data through
the standard Android logging interface. When experimentation completes, the \PhoneLab{} user receives an
archive containing every log message matching their tags generated by all
participating devices.

%\PhoneLab{} users must provide human subjects review documentation, a list of
%log tags to capture (which we describe later in this section), and their
%experimental software---either a link to the Play Store or a patch against
%the current \PhoneLab{} platform source.  Experiments generate data through
%the standard Android logging interface. Log messages generated by \PhoneLab{}
%experiments are captured and uploaded to a central server while the device is
%plugged in and charging. When experimentation completes, the user receives an
%archive containing every log message matching their tags generated by all
%participating devices.

\subsection{Platform and Device}

\PhoneLab{} phones run the popular Google Android open-source smartphone
platform (AOSP). Using an open-source platform for \PhoneLab{} was an obvious
choice for obvious and less-obvious reasons.

The obvious reason is that the AOSP allows \PhoneLab{} users to experiment
with any software component, meeting our goal of providing a powerful
testbed. Modifications to Android services that provide location, access
networks, and manage power can be benchmarked alongside unmodified devices.
Of course, power also creates problems: faulty experiments can render phones
inoperable and threaten participation. As a result, experimentation at the
platform level will require additional pre-deployment testing and interaction
with the \PhoneLab{} team when compared with experiments that only distribute
novel applications or collect data at the application level.

We have also found that using an open-source platform has other, less obvious
benefits. First, the availability of the Android source makes \PhoneLab{}
instrumentation easier even when collecting data from the application level
because it gives a visibility into hidden APIs. For example, our usage
characterization experiment, described in Section~\ref{sec-experiment}, uses
Java reflection to access hidden battery usage APIs.

Second, the AOSP allows us to sign the platform image used by our
participants thus providing application root privileges at run time.
%When the same key is used to sign a software package, that
%application may run as the system user with root privileges. 
Using this feature allows us to distribute and update core \PhoneLab{} experimental
management software via the Play Store while retaining the privileges necessary
to collect logs and perform platform updates.

Finally, we expect that our base \PhoneLab{} platform image will evolve to
meet the needs of the research community. While we have found that Android
already logs a wealth of information about platform operation, there are
places where more information could be exposed or logged in a more
experiment-friendly way. Controlling the platform provides the opportunity to
supplement existing interfaces or add additional logging to make
experimentation and data collection easier.


\subsection{Participants}
When recruiting our first batch of participants, we primarily target freshman
and sophomore SUNY Buffalo (UB) students as well as incoming PhD students. Later on around early September 2012, we also began
to reach out to the professional population at SUNY Buffalo in an effort to
increase the number of potential long-term participants as well as the diversity
of our participant pool.

\begin{table}[t]
\begin{threeparttable}
\begin{tabularx}{\columnwidth}{Xr@{\hspace{0.5in}}Xr}
\multicolumn{4}{c}{\textbf{Affiliation}} \\
\midrule
Freshman & 64 & Masters & 5 \\
Sophomore & 33 & PhD & 53 \\
Junior & 1 & Faculty/Staff & 29 \\
Senior & 1 & None & 5 \\[0.1in]
\multicolumn{4}{c}{\textbf{Gender}} \\
\midrule
Female & 51 & Male & 140 \\[0.1in]
\multicolumn{4}{c}{\textbf{Age}} \\
\midrule
Under 18 & 12 & 30--34 & 15 \\
18--19 & 74 & 35--39 & 6 \\
20--21 & 12 & 40--49 & 13 \\
22--24 & 22 & 50--59 & 7 \\
25--29 & 29 & 60+ & 1 \\
\end{tabularx}
\end{threeparttable}
\caption{\textbf{Demographic breakdown of 191 \PhoneLab{} participants.} Date
ranges are inclusive.}
\label{table-demographics}
\end{table}

In the end, we believe that we were successful in recruiting potential
long-term participants. Table~\ref{table-demographics} describes the
demographic breakdown that we achieved. 

\subsection{Testbed Software}

Experimental configuration, log collection, data upload and platform updates
are performed by the \PhoneLab{} experimental harness, which is installed and
updated through the Google Play Store. By signing it to match the platform
build key it runs with root privileges, necessary to collect logs from all
applications and perform platform updates. Periodically, the experimental
harness retrieves an XML configuration from a central \PhoneLab{} server. The
configuration specifies what background experiments to start or stop, what
data to collect, which server the phone should upload data to and the policy
for when to perform uploads. The \PhoneLab{} harness also uploads status
information to the server during the configuration exchange, including what
versions of various harness components are installed, what experiments are
running and how much data is waiting to be uploaded.

\PhoneLab{} logging and data collection must be unintrusive. If it is not,
either our participants will leave or their usage patterns will be affected.
We believe that we have achieved this goal. First, measured battery usage of
\PhoneLab{} is low. A conservative overhead estimate that includes all of the
applications that run as the shared system user comes to a per-participant
average of 2.4\%. This should be considered a strict maximum. Our policy of
only uploading while the device is plugged and charging eliminates the
overhead of the most power-hungry task.

Second, we have received no major complaints about our the final version of
our \PhoneLab{} experimental harness after we instructed participants to
install it. Given that participants we allowed to use their phone without our
software for several months, we believe that any significant changes in phone
behavior caused by our experimental harness would have been noticed.

\subsection{Safety and Privacy}

\PhoneLab{} is different from many other computer systems testbeds, such as
Emulab~\cite{white:osdi:2002, emulab}, PlanetLab~\cite{peterson:ccr:2003,
planetlab}, MoteLab~\cite{werner-allen:ipsn:2005}, or
OpenCirrus~\cite{avetisyan:computer:2010, opencirrus}: our experiments
involve real people. There are two core requirements regarding our
participants. First, they should use their phone as they normally would,
which motivated the design of unintrusive testbed management software.
Second, and more importantly, they must feel safe and in control while part
of \PhoneLab{}.

To accomplish this, when possible, we leverage several existing safety
mechanisms. First, we require an Institutional Review Board (IRB) to review
each \PhoneLab{} experiment for human subjects compliance. IRB approval or an
official waiver is required before any \PhoneLab{} any experiment can begin.


Finally, we utilize Android's existing safety and privacy mechanisms.
Participants are presented with the typical Android privacy dialog during
experiment installation. Rather than building an alternate distribution
channel or privacy mechanism, we felt it was sufficient and probably better
to use a process participants are familiar with. After installation, if a
participant discovers that an experiment malfunctions or wastes power, they
can uninstall it. If we notice patterns of experimental removal, we will flag
the experiment and notify the researcher.

\input{./figures/logging/table.tex}


\input{./figures/statistics/tag_table/table.tex}

\subsection{Experimental Procedures}

To conclude, we review \PhoneLab{} experimentation from a researcher's
perspective.

First, develop your application locally. Any information logged through the
standard Android logging library can be recorded. In addition, the platform may
already be logging useful information for you. Keep track of all the log tags
you want \PhoneLab{} to capture. Approach your local IRB and receive
experimental approval and upload your application to the Play Store.

Second, upload your list of log tags, IRB letter, and link to your
application on the Play Store through the \PhoneLab{} website. We will
contact you when we begin beta testing and again once your experiment is
ready for the testbed. During beta testing you will be provided with
\PhoneLab{} log output to ensure that your experiment is running properly.

Finally, your experiment will be scheduled. Our goal is to maintain a
medium-sized list of active experiments for our participants: large enough to
make good use of the testbed, but small enough to ensure that each experiment
is picked up by many participants. When your experiment completes, you will
receive a archive with messages matching the tags you selected.
