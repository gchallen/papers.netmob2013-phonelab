\clearpage

\section{The PhoneLab Testbed}
\label{sec-testbed}

\PhoneLab{} was designed to fill a gap in existing smartphone experimentation
capabilities. Specifically, current experimental approaches are forced to
trade off:

\begin{itemize}[nosep]
\vspace*{0.1in}
\item power: permitting the modification of any smartphone software component,
\item scale: facilitating deployment on many devices,
\item and realism: collecting results from a representative set of smartphone users.
\vspace*{0.1in}
\end{itemize}

Distributing experiments through the Play Store or App Store provides realism
but limits research to distributing applications, with scale a possibility
but a challenge due to lack of incentives. Distributing experiments to
graduate students provides power but limits scale and (probably) realism.

\PhoneLab{} achieves power by utilizing Android open-source devices and a
self-signed build which allows us to update any software components; scale by
amortizing recruitment overhead, management burden and incentive costs across
multiple experiments; and realism by recruiting a diverse set of participants
and limiting experimental intrusiveness. We describe the architecture of the
\PhoneLab{} testbed in more detail below.

\subsection{Overview}

\PhoneLab{} currently consists of 190 participants\footnote{We refer to
people carrying \PhoneLab{} phones and participating in experiments as
\PhoneLab{} \textit{participants}, to differentiate them from researchers
running \PhoneLab{} experiments which we call \textit{users}.} using Sprint
Nexus~S~4G smartphones~\cite{FIXME-nexuss4g} running Android 4.1.1 ``Jelly
Bean''~\cite{FIXME-jellybean}. Participants receive discounted voice, data,
and messaging and are instructed to use their \PhoneLab{} phone as their
primary device.

\PhoneLab{} experiments are either distributed through the Play Store or as
platform over-the-air (OTA) updates. Participants are notified of new
experiments and choose whether to participate after reviewing what
information will be collected about them. \PhoneLab{} participants are
\textit{required} to participate in experimentation but \textit{not required}
to participate in any particular experiment. They may uninstall experiments
that they deem too intrusive or that negatively affect their device.

\PhoneLab{} users must provide human subjects review documentation, a list of
log tags to capture, and their experimental software---either a link to the
Play Store or a patch against the current \PhoneLab{} platform source.
Experiments generate data through the standard Android logging interface. Log
messages generated by \PhoneLab{} experiments are captured and uploaded to a
central server while the device is plugged in and charging. When
experimentation completes, the user receives an archive containing every log
message matching their tags generated by all \PhoneLab{} devices that
participated in their experiment.

\subsection{Platform and Device}

\PhoneLab{} phones run the popular Google Android open-source smartphone
platform (AOSP). Using an open-source platfrom for \PhoneLab{} was an obvious
choice for obvious and less-obvious reasons.

The obvious reason is that the AOSP allows \PhoneLab{} users to experiment
with any software component, meeting our goal of providing a powerful
testbed. Modifications to Android services that provide location, access
networks, and manage power can be benchmarked alongside unmodified devices.
Of course, power also creates problems: faulty experiments can render phones
inoperable and threaten participation. As a result, experimentation at the
platform level will require additional pre-deployment testing and interaction
with the \PhoneLab{} team when compared with experiments that only distribute
novel applications or collect data at the application level.

We have also found that using an open-source platform has other, less obvious
benefits. First, the availability of the Android source makes \PhoneLab{}
instrumentation easier even when collecting data from the application level.
Our usage characterization experiment, described in
Section~\ref{sec-experiment}, uses Java introspection facilitated by access
to the Android source to access battery usage information not publicly
exposed by Android 4.1.1.

Second, the AOSP allows us to sign the platform image used by our
participants. When the same key is used to sign a software package, that
application may run as the system user with root privileges. Using this
feature allows us to distribute and update core \PhoneLab{} experimental
management software via the Google Play Store while retaining the privileges
necessary to collect logs and perform platform updates.

Finally, we expect that our base \PhoneLab{} platform image will evolve to
meet the needs of the research community. While we have found that Android
already logs a wealth of information about platform operation, there are
places where more information could be exposed or logged in a more
experiment-friendly way. Controlling the platform provides the opportunity to
supplement existing interfaces or add additional logging to make
experimentation and data collection easier.

We have distributed Nexus S 4G smartphones to our first group of
participants. The Nexus S 4G was first released by Sprint in May, 2011, and
was one of the official AOSP development phones at the time \PhoneLab{}
development began. It features include:

\begin{itemize}[nosep]
\item a 1~GHz ARM Cortex A8 CPU,
\item a PowerVR SGX 540 GPU,
\item 512~MB of RAM with 128~MB reserved for the GPU,
\item 16~GB of NAND Flash divided into 1~GB internal and 15~GB external storage partitions,
\item a 1500 mAh battery,
\item a 4" 480x800 resolution touch screen display, and
\item 3G/4G (WiMax)\footnote{Sprint has not deployed 4G WiMax in the Buffalo
area}, 802.11 b/g/n WiFi, Bluetooth, NFC (near-field communication) and USB
networking interfaces.
\end{itemize}

While we expect to receive yearly phone upgrades and will distribute more
up-to-date device to our second group of participants, we anticipate that the
prohibitive cost of the newest smartphones will prevent us from ever
deploying them on \PhoneLab{}.

\subsection{Participants}

Recruiting a large number of \PhoneLab{} participants requires effective
incentives. In their first year of \PhoneLab{} participation, voice data and
messaging are free with funding provided by the National Science Foundation
(NSF). This free year of service plays a major role in our recruiting
efforts. In subsequent years, participants pay a deeply discounted \$45 per
month rate for unlimited data and messaging through a deal negotiated with
Sprint. Sprint has proved to be an ideal partner for the \PhoneLab{} project,
both helpful with testbed logistics and still willing to provide unlimited
data plans to subscribers.

Because participants may leave at any time, the front-loaded cost structure
of our incentives makes it most efficient to recruit participants who will be
able to continue as part of \PhoneLab{} for multiple years. While we
anticipate that some of our first group of participants will leave after a
single year, they will help us more effectively recruit long-term
participants during future years. On the other hard, long-term participants
allow us to continue to amortize the costs of the first free year and of
recruitment itself while providing us with a stable set of participants
comfortable being a part of \PhoneLab{} experimentation.

When recruiting our first batch of participants, we initially intended to
target freshman and sophomore students studying computer science and
engineering. We found, however, that many of these students already owned
smartphones. At this point we broadened our first- and second-year student
recruitment across all disciplines as well as incoming PhD students. SUNY
Buffalo has a large international graduate student community, and many of
these students arrive on campus without phones or phone contracts, making
them ideal multi-year \PhoneLab{} participants.

In the end, as Table~\ref{table-demographics} illustrates, the majority of
our p
