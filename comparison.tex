\begin{table*}[t]
\begin{tabularx}{\textwidth}{Xcccccc}
& {\normalsize{\textbf{Scale}}} &
{\normalsize{\textbf{Realism}}} &
{\normalsize{\textbf{Locality}}} &
{\normalsize{\textbf{Relevance}}} &
{\normalsize{\textbf{Power}}} \\
\toprule

{\large \PhoneLab{}}
& $\blacksquare$ & $\blacksquare$ & $\blacksquare$ & $\blacksquare$ & $\blacksquare$ \\
\toprule

LiveLabs &
$\blacksquare$ & $\blacksquare$ & $\blacksquare$ & $\blacksquare$ & & \\
\midrule

One-off Studies &
& & $\blacksquare$ & $\blacksquare$ & $\blacksquare$ \\
\midrule

Application Marketplaces &
$\blacksquare$ & $\blacksquare$ & & $\blacksquare$ & \\
\midrule

Large Traces &
$\blacksquare$ & $\blacksquare$ & $\blacksquare$ & & \\
\bottomrule

\end{tabularx}
\caption{\textbf{Smartphone experimentation comparison.} Only \PhoneLab{}
provides all necessary features.}
\label{tab:comparison}
\end{table*}

\section{Smartphone Experimentation}
\label{sec-comparison}

\PhoneLab{} provides the features necessary for smartphone research---power,
scale, realism, locality, and relevance:

\begin{itemize}[nosep,leftmargin=*]
\vspace*{0.08in}
\item {\bf Scale:} \PhoneLab{} will grow to 700 participants, already
recruited to participate in experiments.
\item {\bf Power:} \PhoneLab{} allows application-level experiments as well
as platform-level, i.e., Android kernel, middleware, libraries, and Dalvik
virtual machine.
\item {\bf Realism:} Participants use the phones as their primary device.
\item {\bf Locality:} Most participants live in Buffalo near SUNY campuses,
enabling research requiring device-to-device interaction.
\item {\bf Relevance:} \PhoneLab{} allows researchers to stop relying on
out-of-date datasets. Instead, new data can be collected in the most
appropriate way for the experiment.
\vspace*{0.08in}
\end{itemize}

We believe that \PhoneLab{} is the only smartphone testbed providing all of
these features together. Table~\ref{tab:comparison} summarizes \PhoneLab{}'s
features compared to other approaches, which we discuss individually below.

\subsection{LiveLabs}

LiveLabs~\cite{livelabs} is the most similar testbed to \PhoneLab{}. While it
provides many desirable features, it lacks the ability to perform platform
experimentation. LiveLabs will eventually have three sites---a mall, a
campus, and a theme park---that are instrumented with custom wireless
infrastructure. Their goal is to scale to \num{25000} participants, but
currently only have tens of participants connected. LiveLabs plans to provide
public interfaces to collect processed location and activity information.
Their aim seems largely to support commercial and marketing-driven location
and consumer analytics, rather than smartphone systems experimentation.
However, given that the testbed is still being built, its features may
change.

\subsection{One-off Studies}

The standard evaluation methodology when using real devices is to recruit a
small number of volunteers composed of fellow researchers and acquaintances.
Clearly the scale that can be achieved this way is limited, but this approach
also lacks realism. A small number of personal contacts is unlikely to
produce a representative sample, and users in a short-term study are unlikely
to be willing to use a new phone as their primary device. In comparison,
\PhoneLab{} amortizes the recruitment and maintenance costs across multiple
experiments, and provides repeated access to the same participants so that
findings can be verified and competing approaches compared.

\subsection{Application Marketplaces}

A few research projects~\cite{huang:mobisys:2010, zhang:codes:2010} have
demonstrated that it is possible to reach large numbers of users by using
online application stores such as the Google Play Store or Apple App Store.
Assuming participants can be incentivized to participate in experiments, this
methodology still lacks power and locality. It is unlikely that applications
distributed via online marketplaces will be downloaded and used by large
numbers of co-located participants. In addition, these stores only
distributed applications and cannot be used for experimentation with core
system components.

\subsection{Large Traces}

Research evaluations that require large-scale measurement data can be
challenging, as large-scale measurement requires a combination of scale,
realism, and relevance. In order to conduct large-scale measurement studies,
researchers have resorted to arduous methods such as obtaining and analyzing
service providers' network-level data sets~\cite{xu:imc:2011,
trestian:imc:2009, trestian:ton:2012}, incentivizing a large number of
participants themselves~\cite{falaki:mobisys:2010}, or recruiting volunteers
through a long period of publicizing the project~\cite{shye:micro:2009}.
Although it is possible to obtain large-scale traces using these methods, the
experiments are not repeatable and the traces quickly become obsolete.

Our experiment case study described in Section~\ref{sec-experiment}
demonstrates how \PhoneLab{} greatly simplifies trace collection, while
providing access to participants and allowing experiments to be repeated. The
inherent difficulty of large-scale trace collection is precisely why we use our own
data collection and analysis to demonstrate the power of \PhoneLab{}.
