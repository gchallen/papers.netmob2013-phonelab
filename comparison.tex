\begin{table*}[t]
\begin{tabularx}{\textwidth}{Xcccccc}
& {\normalsize{\textbf{Scale}}} &
{\normalsize{\textbf{Realism}}} &
{\normalsize{\textbf{Locality}}} &
{\normalsize{\textbf{Relevance}}} &
{\normalsize{\textbf{Power}}} \\
\toprule

{\large \PhoneLab{}}
& $\blacksquare$ & $\blacksquare$ & $\blacksquare$ & $\blacksquare$ & $\blacksquare$ \\
\toprule

LiveLabs &
$\blacksquare$ & $\blacksquare$ & $\blacksquare$ & $\blacksquare$ & & \\
\midrule

One-off Studies &
& & $\blacksquare$ & $\blacksquare$ & $\blacksquare$ \\
\midrule

Application Marketplaces &
$\blacksquare$ & $\blacksquare$ & & $\blacksquare$ & \\
\midrule

Large Traces &
$\blacksquare$ & $\blacksquare$ & $\blacksquare$ & & \\
\bottomrule

\end{tabularx}
\caption{\textbf{Smartphone experimentation comparison.} Only \PhoneLab{}
provides all necessary features.}
\label{tab:comparison}
\end{table*}

\section{Smartphone Experimentation}
\label{sec-comparison}

\PhoneLab{} provides the features necessary for smartphone research---power,
scale, realism, locality, and relevance:

\begin{itemize}[nosep,leftmargin=*]
\vspace*{0.08in}
\item {\bf Scale:} \PhoneLab{} will grow to 700 participants, already
recruited to participate in experiments.
\item {\bf Power:} \PhoneLab{} allows application-level experiments as well
as platform-level, i.e., Android kernel, middleware, libraries, and Dalvik
virtual machine.
\item {\bf Realism:} Participants use the phones as their primary device.
\item {\bf Locality:} Most participants live in Buffalo near SUNY campuses,
enabling research requiring device-to-device interaction.
\item {\bf Relevance:} \PhoneLab{} allows researchers to stop relying on
out-of-date datasets. Instead, new data can be collected in the most
appropriate way for the experiment.
\vspace*{0.08in}
\end{itemize}

We believe that \PhoneLab{} is the only smartphone testbed providing all of
these features together. Table~\ref{tab:comparison} summarizes \PhoneLab{}'s
features compared to other approaches.

