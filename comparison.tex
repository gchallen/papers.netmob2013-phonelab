\begin{table*}[t]
\begin{tabularx}{\textwidth}{Xcccccc}
& {\normalsize{\textbf{Scale}}} &
{\normalsize{\textbf{Realism}}} &
{\normalsize{\textbf{Locality}}} &
{\normalsize{\textbf{Timeliness}}} &
{\normalsize{\textbf{Power}}} \\
\toprule

{\large \PhoneLab{}}
& $\blacksquare$ & $\blacksquare$ & $\blacksquare$ & $\blacksquare$ & $\blacksquare$ \\
\toprule

LiveLabs &
$\blacksquare$ & $\blacksquare$ & $\blacksquare$ & $\blacksquare$ & & \\
\midrule

One-off Studies &
& & $\blacksquare$ & $\blacksquare$ & $\blacksquare$ \\
\midrule

Application Marketplaces &
$\blacksquare$ & $\blacksquare$ & & & \\
\midrule

Large Traces &
$\blacksquare$ & $\blacksquare$ & $\blacksquare$ & & \\

\end{tabularx}
\caption{\textbf{Smartphone experimentation comparison.} Only \PhoneLab{}
provides all necessary features.}
\label{tab:comparison}
\end{table*}

\section{Other Experimental Approaches}
\label{sec-comparison}

\PhoneLab{} provides features necessary for smartphone research---power,
scale, realism, locality, and timeliness. We define these features as:

\begin{itemize}[nosep]
\item {\bf Scale:} \PhoneLab{} will grow to 700 participants, already
recruited to participate in smartphone experiments.
\item {\bf Power:} \PhoneLab{} allows application-level experiments as well
as platform-level, i.e., Android kernel, middleware, libraries, and Dalvik
virtual machine.
\item {\bf Realism:} Participants use the phones as their primary device.
\item {\bf Locality:} Most participants live in Buffalo near SUNY campuses,
enabling research requiring device-to-device interaction.
\item {\bf Timeliness:} \PhoneLab{} allows researchers to quit relying on
out-of-date datasets. Instead, new data can be collected and customized to
the experiment being performed.
\end{itemize}

We believe that \PhoneLab{} is the only smartphone testbed providing all of
these features together. Table~\ref{tab:comparison} summarizes \PhoneLab{}'s
features compared to other approaches, which we discuss individually below.

LiveLabs~\cite{livelabs} is the most similar testbed to \PhoneLab{}. While it
provides many desirable features, it lacks the ability to perform platform
experimentation. LiveLabs will eventually have three sites---a mall, a
campus, and a theme park---that are instrumented with custom wireless
infrastructure. Their goal is to scale to \num{25000} participants, but
currently only have tens of participants connected. LiveLabs plans to provide
public interfaces to collect processed location and activity information.
Their aim seems largely to support commercial and marketing-driven location
and consumer analytics, rather than smartphone systems experimentation.
However, given that the testbed is still being built, its features may
change.

The standard evaluation methodology when using real devices is to recruit a
small number of volunteers composed of fellow researchers and acquaintances.
Clearly the scale that can be achieved this way is limited, but this approach
also lacks realism. A small number of personal contacts is unlikely to
produce a representative sample, and users in a short-term study are unlikely
to be willing to use a new phone as their primary device. In comparison,
\PhoneLab{} amortizes the recruitment and maintenance costs across multiple
experiments, and provides repeated access to the same participants so that
finding can be verified and competing approaches compared.

A few research tools~\cite{huang:mobisys:2010, zhang:codes:2010} have
demonstrated that it is possible to reach large numbers of users by deploying
the tools on online application stores such as the Google Play Store or Apple
App Store. Assuming participants can be incentivized to participate in
experiments, this methodology still lacks power and locality. It is unlikely
that applications distributed via online marketplaces will be downloaded and
used by large numbers of co-located participants. In addition, these stores
only distributed applications and cannot be used for experimentation with
core system components.

{\bf Large-Scale Traces:} A particularly challenging kind of research evaluation
is the one that requires large-scale measurement data. Smartphone usage studies
and battery characteristics studies are some of the examples that belong to this
category. This is because large-scale measurement requires a combination of many
features such as scale, realism, timeliness, and participant control. This is
precisely the reason why we use our own data collection and analysis as a
showcase to demonstrate the power of \PhoneLab{} in this paper.

In order to conduct large-scale measurement studies, researchers have
resorted to arduous methods such as obtaining and analyzing service
providers' network-level data sets~\cite{xu:imc:2011, trestian:imc:2009,
trestian:ton:2012}, incentivizing a large number of participants
themselves~\cite{falaki:mobisys:2010}, recruiting volunteers through a long
period of publicizing the project~\cite{shye:micro:2009}, etc. Although it is
possible to obtain large-scale traces using these methods, they are nearly
impossible to repeat. As a result, the traces age rapidly and quickly become
obsolete. Our experiment case study described in Section~\ref{sec-experiment}
demonstrates how \PhoneLab{} greatly simplifies trace collection, while
providing access to participants and allowing experiments to be repeated.
