\begin{table*}[t]
\centering
\begin{tabular}{|c|c|c|c|c|c|c|}
\hline
& Scale & Realism & Locality & Timeliness & Participant Control & Platform
Exp.\\
\hline
{\bf \PhoneLab{}} & {\bf O} & {\bf O} & {\bf O} & {\bf O} & {\bf O} & {\bf O}\\
LiveLabs{} & O & O & O & O & O & X\\
Small-Scale Volunteers & X & X & O & O & X & O\\
Online Application Stores & O & O & X & O & X & X\\
Large-Scale Traces & O & O & O & X & X & O\\
\hline
\end{tabular}
\caption{Feature Comparison}
\label{tab:comparison}
\end{table*}

\section{Comparison of Approaches}
\label{sec:comparison}

\PhoneLab{} provides a comprehensive set of features necessary for smartphone
research evaluation---scale, realism, locality, timeliness, participant control,
and platform experimentation. To the best of our knowledge, \PhoneLab{} is the
only smartphone testbed that plans to provide these features together. Existing
approaches typically lack one or multiple of these features as we discuss below.

{\bf Features of \PhoneLab{}:} In order to facilitate our discussion on existing
approaches, we first present the features that \PhoneLab{} provides:

\begin{itemize}[nosep]
\item {\bf Scale:} \PhoneLab{} is schedule to grow up to 700 participants.
\item {\bf Realism:} \PhoneLab{} participants use the phones as their primary
phones.
\item {\bf Locality:} \PhoneLab{} participants are largely co-located in or near
SUNY Buffalo campuses, enabling the evaluation of research that needs local
communication.
\item {\bf Timeliness:} \PhoneLab{} experimenters do not need to resort to
obsolete data sets collected once; \PhoneLab{} allows access to a large number
of smartphone users when an experimenter needs them.
\item {\bf Participant Control:} \PhoneLab{} participants are pre-recruited and
incentivized already; experimenters do not need to go through this process.
\item {\bf Platform Experimentation:} \PhoneLab{} allows application-level
experiments as well as platform-level, i.e., Android kernel, middleware,
libraries, and Davik virtual machine.
\end{itemize}

Table~\ref{tab:comparison} summarizes \PhoneLab{}'s features compared to other
approaches.

{\bf LiveLabs:} LiveLabs~\cite{livelabs} is the most similar testbed that we are
aware of. Its target feature set is similar to that of \PhoneLab{} (i.e., scale,
realism, locality, timeliness, and participant control), but lacks the ability
to perform platform experimentation. LiveLabs will eventually have three
sites---a mall, a campus, and a theme park---that are instrumented with custom
wireless infrastructure; and it will provide APIs to collect location and
activity information from the participants. Since it is a work in progress, the
exact features and operational details are not yet documented well.

{\bf Small-Scale Volunteers:} A relatively effortless evaluation methodology
with real devices is perhaps to use a small number of volunteers composed of
research group members and acquaintances. This methodology obviously lacks
scale, but also suffers from the question of realism; with only a small number
of people recruited through personal contacts, it is difficult to justify that
they are representative. Using an open-access testbed such as \PhoneLab{}
removes the burden of justifying a particular evaluation methodology from
individual researchers.

{\bf Online Application Stores:} A few research tools~\cite{huang:mobisys:2010,
zhang:codes:2010} have demonstrated that it is possible to reach a large
population of users by deploying the tools on online application stores (e.g.,
Google Play Store and Apple App Store). However, this methodology lacks a few
features such as locality, participant control, and platform experimentation; it
is not guaranteed that the tools will be downloaded and used by many people; it
does not give any control over locality as online stores are available
world-wide; it does not allow platform experimentation since online stores
distribute applications only. This methodology is suitable when these are not
the criteria for evaluation.

{\bf Large-Scale Traces:} A particularly challenging kind of research evaluation
is the one that requires large-scale measurement data. Smartphone usage studies
and battery characteristics studies are some of the examples that belong to this
category. This is because large-scale measurement requires a combination of many
features such as scale, realism, timeliness, and participant control. This is
precisely the reason why we use our own data collection and analysis as a
showcase to demonstrate the power of \PhoneLab{} in this paper.

In order to conduct large-scale measurement studies, researchers have resorted
to arduous methods such as obtaining and analyzing service providers'
network-level data sets~\cite{xu:imc:2011, trestian:imc:2009,
trestian:ton:2012}, incentivizing a large number of participants
themselves~\cite{falaki:mobisys:2010}, recruiting volunteers through a long
period of publicizing the project~\cite{shye:micro:2009}, etc. Although it is
possible to obtain large-scale traces using these methods, they are
prohibitively expensive to repeat. As a result, the traces quickly become
obsolete and lose the timeliness of data. Our case study demonstrates that
\PhoneLab{} greatly simplifies this process. With \PhoneLab{}, {\it any}
researcher can use {\it real, up-to-date} smartphone data from a large pool of
participants {\it according to their needs when they need it}.
