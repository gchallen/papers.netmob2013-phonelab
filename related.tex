\section{Related Work}
\label{sec:related}

We categorize our related work into three areas---testbeds, smartphone
measurement studies, and smartphone measurement tools. We discuss each category
below.

{\bf Testbeds:} In many research domains, a testbed is typically tailored
to the needs of its target domain and proven to be a valuable tool. For example,
recognizing the difficulty for conducting large-scale, realistic Internet
research, PlanetLab~\cite{peterson:ccr:2003, planetlab} operates more than 1,000
machines world-wide. Emulab~\cite{white:osdi:2002, emulab} is also designed for
large-scale networking research but provides emulated network environments for
controlled, repeatable experiments.  MoteLab~\cite{werner-allen:ipsn:2005} is a
sensor network testbed deployed in a building at Harvard, enabling realistic
sensor network experiments.  ORBIT~\cite{raychaudhuri:tridentcom:2005} takes a
two-tier approach allowing emulated experiments as well as real deployments,
targeting reproducibility and realism at the same time. Likewise, \PhoneLab{} is
tailored to the needs of smartphone research with its focus on scale and
realism.

To the best of our knowledge, LiveLabs~\cite{livelabs} is the only smartphone
testbed except \PhoneLab{} that promises access to a large number of
participants. Since it is a work in progress, the exact features and operational
details are not yet documented well. According to their website, it will
eventually have three sites---a mall, a campus, and a theme park---that are
instrumented with custom wireless infrastructure. It will collect location and
activity information from the participants.

{\bf Smartphone Usage Measurement Studies:} A number of studies have looked at
smartphone usage from various perspectives. Falaki et
al.~\cite{falaki:mobisys:2010} are among the first ones to study smartphone
usage. Their central finding is that in many of the metrics they studied, there
was significant diversity without a clear pattern; the metrics include the mean
interaction length, the mean number of applications used, the mean amount of
traffic, etc. Xu et al.~\cite{xu:imc:2011} uses a network-level trace to analyze
smartphone application usage. The key findings are that smartphone users use
many regional applications such as local news apps; certain applications are
installed together; and mobility patterns affect the types of applications
used. Shye et al.~\cite{shye:micro:2009} studies how power consumption is
distributed over different hardware components. 

serendipity~\cite{trestian:imc:2009}
dropzones~\cite{trestian:ton:2012}
storage~\cite{kim:fast:2012}
power usage~\cite{shye:micro:2009}
\XXXnote{stevko: Say something about our findings in comparison to these
studies.}

{\bf Smartphone Measurement Tools:}
