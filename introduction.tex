\section{Introduction}
\label{sec-introduction}

\sloppypar{Smartphone usage is exploding. Google reports 1.3~M Android device
activations per day in September, 2013~\cite{google-Sep2012-activations},
while IDC projects that 224~M smartphone units will ship worldwide in 2013
Q4, a 40\% increase over 2012 Q4~\cite{idc-smartphone-growth}. Taken as a
whole, the growing network of smartphone devices represents the largest and
most pervasive distributed system in history.}

Meanwhile, the scale of smartphone experimentation is not keeping pace. A
small survey of MobiSys'12 papers reveals that when smartphone evaluations
use real devices, they use small numbers of phones---3, 12, or
20~\cite{nowar-mobisys12,comon-mobisys12,caching-mobisys12}. Other
experiments use simulations driven by small, old, or synthesized data
sets~\cite{falcon-mobisys12,ace-mobisys12,humanmobility-mobisys12}. In either
case, large-scale results from real users would be more compelling. While
multiple factors---including recruitment, human subjects compliance, and data
collection---make large-scale smartphone challenging, harnessing the growth
of smartphones requires evaluating new ideas at scale.

\vfill\eject

We present \PhoneLab{}, a large programmable smartphone testbed enabling
smartphone research at scales currently impractical. \PhoneLab{} provides
access to a large and stable set of participants incentivized to participate
in smartphone experimentation. From 191~participants in 2012, \PhoneLab{}
will grow to over 700 participants in 2014. By exploiting locality,
\PhoneLab{} increases the density and interaction rate between participants,
facilitating the evaluation of phone-to-phone protocols and crowd-sourcing
algorithms.

By utilizing the Android open-source smartphone platform, \PhoneLab{} enables
research above and below the platform interface. Researchers can distribute
new interactive applications or non-interactive data loggers, but can also
change core Android platform and kernel components, allowing \PhoneLab{} to
host systems experiments impossible to distribute through the Play Store.

We make two contributions in this paper. First, we introduce \PhoneLab{}.
After comparing against other approaches in Section~\ref{sec-comparison},
Section~\ref{sec-testbed} describes the testbed design and implementation,
presents demographic data on our participants, and explains experimental
procedures. Section~\ref{sec-logging} describes the Android logging framework
and the visibility it provides into the Android platform.

Second, we demonstrate that \PhoneLab{} is powerful and usable.
Section~\ref{sec-experiment} describes a usage measurement experiment run by
115 \PhoneLab{} participants for 21 days. Rather than attempting a
comprehensive analysis of the dataset, we use it to highlight the power of
\PhoneLab{} and breadth of research it supports. To do so,
Section~\ref{sec-usage} presents five results on:

\begin{itemize}[nosep]
\vspace*{0.08in}
\item overall battery usage (Section~\ref{subsec-batteryoverview}),
\item opportunistic charging (Section~\ref{subsec-opportunistic}),
\item 3G to WiFi transitions (Section~\ref{subsec-networktransitions}),
\item application usage patterns (Section~\ref{subsec-apptransitions}), and
\item location data sharing (Section~\ref{subsec-locationsharing}).
\vspace*{0.08in}
\end{itemize}

For each, we first present our data and then describe how to use \PhoneLab{}
to perform further investigation. We hope that these examples will encourage
\PhoneLab{} use by the mobile systems community.
