\section{Introduction}
\label{sec-introduction}

Smartphone usage is exploding. Google reports 1.3~M Android device activations
per day in September, 2013~\cite{google-Sep2012-activations}. IDC projects that
224~M smartphone units will ship worldwide in 2013 Q4, a 40\% increase over
2012 Q4~\cite{idc-smartphone-growth}. Taken as a whole, the growing network of
smartphone devices represents the largest and most pervasive distributed system
in history.

Meanwhile, the scale of smartphone experimentation is lagging. A survey of
MobiSys'12 papers shows reveals that when smartphone evaluations use real
devices, this use small numbers---3, 12, or 20
phones~\cite{nowar-mobisys12,comon-mobisys12,caching-mobisys12}. Other
experiments use simulations driven by small, old, or synthesized data
sets~\cite{falcon-mobisys12,ace-mobisys12,humanmobility-mobisys12}. In either
case, large-scale results from representative users are more compelling.
While multiple factors---including recruitment, human subjections compliance,
and data collection---make large-scale smartphone challenging, harnessing the
growth of smartphones requires evaluating ideas at scale.

We present \PhoneLab{}, an open-access programmable smartphone testbed
designed to enable smartphone research at scales not currently practical.
\PhoneLab{} provides access to a large and stable set of participants,
already incentivized to participate in smartphone experimentation. Currently
comprising 191~participants, \PhoneLab{} is scheduled to grow to over 700
devices by 2014. By exploiting locality, \PhoneLab{} increases the density
and interaction rate between participating smartphones, which we expect to be
important for evaluating new research on phone-to-phone protocols and
crowdsourcing algorithms.

By utilizing the Android open-source smartphone platform, \PhoneLab{} enables
research both above and below the application-platform interface. Researchers
can distribute new interactive applications or custom non-interactive data
loggers, but they can also change core Android platform and kernel
components, making \PhoneLab{} ideal for platform experiments impossible to
distribute through the Play Store (Android Market).

We make two core contributions in this paper. First, we introduce
\PhoneLab{}, describe the testbed and our participants, and explain
experimental procedures. We also describe the Android logging framework and
discuss the visibility it provides into the inner workings of the Android
platform.

% 02 Dec 2012 : GWA : run on ... phones is
% ./figures/statistics/lib.py % --experiment_count

% 02 Dec 2012 : GWA : for ... days is
% ./figures/statistics/lib.py % --experiment_length_days
% For initial abstract submission this number is estimated.

Second, we present results from a broad usage characterization experiment run
on \PhoneLab{} by 88 participants for 24 days and yielding \XXXnote{GWA} log
messages. We focus our analyses on the patterns that emerge when a large
number of participants are studied over a long period of time. \XXXnote{GWA:
Put three of four top-level findings here.}

The rest of this paper is structured as follows...
