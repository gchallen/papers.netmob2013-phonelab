\section{Introduction}
\label{sec-introduction}

Smartphone usage is exploding. Google reports 1.3~M Android device activations
per day in September, 2013~\cite{google-Sep2012-activations}. IDC projects that
224~M smartphone units will ship worldwide in 2013 Q4, a 40\% increase over
2012 Q4~\cite{idc-smartphone-growth}. Taken as a whole, the growing network of
smartphone devices represents the largest and most pervasive distributed system
in history.

Meanwhile, the scale of smartphone experimentation is lagging. A survey of
MobiSys'12 papers shows reveals that when smartphone evaluations use real
devices, they use small numbers---3, 12, or 20
phones~\cite{nowar-mobisys12,comon-mobisys12,caching-mobisys12}. Other
experiments use simulations driven by small, old, or synthesized data
sets~\cite{falcon-mobisys12,ace-mobisys12,humanmobility-mobisys12}. In either
case, large-scale results from representative users would be more compelling.
While multiple factors---including recruitment, human subjects compliance,
and data collection---make large-scale smartphone challenging, harnessing the
growth of smartphones requires evaluating ideas at scale.

We present \PhoneLab{}, a large programmable smartphone testbed designed to
enable smartphone research at scales currently impractical. \PhoneLab{}
provides access to a large and stable set of participants incentivized to
participate in smartphone experimentation. Currently comprising
191~participants, \PhoneLab{} will recruit over 700 participants by 2014. By
exploiting locality, \PhoneLab{} increases the density and interaction rate
between participats, important for evaluating new research on phone-to-phone
protocols and crowdsourcing algorithms.

By utilizing the Android open-source smartphone platform, \PhoneLab{} enables
research both above and below the application-platform interface. Researchers
can distribute new interactive applications or non-interactive data loggers,
but can also change core Android platform and kernel components, allowing
\PhoneLab{} to host experiments impossible to distribute through the Play
Store.

We make two contributions in this paper. First, we introduce \PhoneLab{}.
Section~\ref{sec-testbed} introduces describe the testbed design and
implementation, presents demographic data on our participants, and explains
experimental procedures. Section~\ref{sec-logging} describes the Android
logging framework and the visibility it provides into the Android platform.

% 02 Dec 2012 : GWA : run on ... phones is
% ./figures/statistics/lib.py % --experiment_count

% 02 Dec 2012 : GWA : for ... days is
% ./figures/statistics/lib.py % --experiment_length_days
% For initial abstract submission this number is estimated.

Second, we demonstrate that \PhoneLab{} is powerful and usable.
Section~\ref{sec-experiment} describes a usage characterization experiment
run by 88 \PhoneLab{} participants for 21 days. Rather than attempting a
comprehensive analysis of the dataset, we use it to highlight the power of
\PhoneLab{} and breadth of research it supports. To do so,
Section~\ref{sec-usage} presents six results on

\begin{itemize}[nosep]
\vspace*{0.08in}
\item overall battery usage (Section~\ref{subsec-batteryoverview}),
\item opportunistic charging (Section~\ref{subsec-opportunistic}),
\item user energy awareness (Section~\ref{subsec-awareness},
\item 3G to WiFi transitions (Section~\ref{subsec-networktransitions}),
\item application usage patterns (Section~\ref{subsec-apptransitions}), and
\item location data sharing (Section~\ref{subsec-locationsharing}).
\vspace*{0.08in}
\end{itemize}

For each, we first present our data and then describe how to use \PhoneLab{}
perform further investigation. We hope that these examples will encourage
\PhoneLab{} use by the mobile systems community.
