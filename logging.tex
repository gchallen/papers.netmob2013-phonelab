\section{Android Instrumentation}
\label{sec-logging}

This section takes a deeper look at how a \PhoneLab{} experiment can examine the
state of a phone, monitor the behavior, and log it. We first discuss the
techniques for monitoring the behavior of a phone. We then discuss Android's
logging mechanism that we leverage for data collection.

\begin{table}[t]
\begin{threeparttable}
{\small
\begin{tabularx}{\columnwidth}{Xrr}
\multicolumn{1}{c}{\normalsize{\textbf{Tag Name}}} & 
\multicolumn{1}{c}{\normalsize{\textbf{Tag Count}}} & 
\multicolumn{1}{c}{\normalsize{\textbf{\%}}} \\
\toprule
\texttt{ActivityManager} & \num{96251731} & 13.7 \\
\texttt{dalvikvm} & \num{92565828} & 13.1 \\
\texttt{ConnectivityService} & \num{19195475} & 2.7 \\
\texttt{ActivityThread} & \num{17447815} & 2.5 \\
\texttt{PhoneStatusBar} & \num{13823998} & 2.0 \\
\texttt{SizeAdaptiveLayout} & \num{9857534} & 1.4 \\
\texttt{wpa\_supplicant} & \num{9279597} & 1.3 \\
\texttt{System.err} & \num{8141399} & 1.2 \\
\texttt{SAN\_SERVICE} & \num{7530577} & 1.1 \\
\texttt{LocationManagerService} & \num{6640001} & 0.9 \\
\texttt{DexLibLoader} & \num{5438086} & 0.8 \\
\texttt{SecCamera} & \num{5436968} & 0.8 \\
\texttt{HeartbeatService} & \num{4871085} & 0.7 \\
\texttt{Beautiful Widgets(4120000)} & \num{4692578} & 0.7 \\
\texttt{AudioCache} & \num{4447544} & 0.6 \\
\texttt{k9} & \num{4330848} & 0.6 \\
\texttt{SensorActivatorService} & \num{4177370} & 0.6 \\
\texttt{ThrottleService} & \num{4121301} & 0.6 \\
\texttt{VoldCmdListener} & \num{4014302} & 0.6 \\
\texttt{WindowManager} & \num{3948168} & 0.6 \\
\texttt{AudioHardware} & \num{3913724} & 0.6 \\
\end{tabularx}
}
\caption{Top 20 log tags generated by Android. \textnormal{\PhoneLab{} has
collected \num{704216410} log messages, of \num{7556} different types. Tags generated
by \PhoneLab{} tools and our usage experiment are ommitted.}}
\label{table-logtags}
\end{threeparttable}
\end{table}


\subsection{Monitoring Techniques}

A \PhoneLab{} user can primarily use three techniques to monitor the behavior of
a phone---subscribing to Android's {\it intents}, periodic logging, and Java
reflection. We describe each technique in detail below.

{\bf Intents:} The first technique is subscribing to Android's intents and
logging when they are delivered. An intent is a messaging object that Android
uses for inter-process communication. It can be either pre-defined by the system
or custom-defined by a developer.

There are two types of intents---unicast intents and broadcast intents. A
unicast intent sends an operation to an application. For example, when an
application is started, Android sends \texttt{android.intent.action.MAIN} intent
to the application to signal that the application should execute the main entry
point. On the other hand, a broadcast intent announces that an event has
happened. For example, when battery is low, Android broadcasts
\texttt{android.intent.action.BATTERY\_LOW} intent.

A \PhoneLab{} user can subscribe to system-defined broadcast intents to monitor
the activities of a phone. Android has an extensive list of broadcast intents
that an application can subscribe to such as application install and uninstall
events, 3G and WiFi signal strength change events, call placed and answered
events, etc. Our own logging tool described in Section~\ref{sec-experiment}, we
subscribe to some of these broadcast intents and record them.

{\bf Periodic Tasks:} The second technique for monitoring the behavior of a
phone is to run a periodic task that collects necessary information and logs it.
Android provides an API (\texttt{android.app.AlarmManager}) that periodically
broadcasts registered intents. An application can create a custom intent,
register it with the system using this API, and collect necessary information
whenever the intent gets delivered. Our logging tool also uses this technique to
periodically collect usage information such as CPU utilization, network traffic
sent and received, battery consumption, etc.

{\bf Java Reflection:} The third monitoring technique is to use Java reflection
to access hidden APIs of Android. Typically, this is not a recommended practice
because hidden APIs can change any time and break the application that uses
them. However, if used carefully, it can be a useful tool to collect information
not available otherwise. Battery statistics is a good example because there is
no public Android API that gives access to the information. However, this
information is still available through hidden APIs that Android's Settings
application uses to display battery statistics. Using Java reflection, any
application can access these APIs. In fact, our logging tool uses this method to
access battery statistics as we describe in Section~\ref{sec-experiment}.

\subsection{Logging}

Since logging and data collection is a common activity for most experiments, we
simplify this process by providing a mechanism that \PhoneLab{} users can use.
Our mechanism leverages the standard Android logging API
(\texttt{android.util.Log}) for logging, collects everything that gets logged
with the API, and uploads the logs to our back-end server.

Collecting logs using the standard API has several benefits. First, any Android
developer is familiar with the logging API because it is frequently used for
debugging purposes. Second, its log format---({\it tag}, {\it
message})---simplifies the process of identifying log-to-experiment mappings; a
\PhoneLab{} user only needs to provide a list of log tags that the user is
interested in and we can provide all the logs that match those tags. Third, the
logs are a rich source of information that provides extra visibility into how a
phone behaves because many system components and applications log various kinds
of information using the API.

For example, \texttt{ConnectivityService} is a system service that manages
mobile and WiFi connectivity and logs connection status; \texttt{SurfaceFlinger}
is a graphics engine abstraction on Android and logs screen unlock and lock
events; \texttt{ActivityManager} is another system service that logs the screen
state; GPS hardware status is logged by \texttt{LocationManagerService}. By
examining these logs together with the logs produced by experiments, \PhoneLab{}
users can gain extra visibility into the state of each phone.
